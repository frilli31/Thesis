% !TEX encoding = UTF-8
% !TEX TS-program = pdflatex
% !TEX root = ../tesi.tex
% !TEX spellcheck = it-IT

%**************************************************************
\chapter{L'Azienda}
\label{cap:azienda}

\intro{Questo capitolo descrive la realtà aziendale e il suo ambito di interesse}

%**************************************************************

\section{Chi è VIC} % DONE
	\paragraph{} 
	\vic{}\cite{site:vic} offre servizi di ispezione delle merci in tutto il mondo per individuare eventuali non conformità delle merci al fine di tutelare i propri i clienti e ridurre al minimo le perdite.
	
	VIC nasce a Venezia nel 2006 ed ora è una delle più grandi compagnie del mercato globale in quanto 
	è presente con 22 diverse aziende e più di 50 laboratori in tutti i più grandi snodi di scambio del mondo.
	
	\paragraph{} La chiave del successo di \vic{} è l'attenzione che pone alla riduzione del tempo tra ispezione e reporting al cliente. 
	Questo è stato possibile grazie all'automazione e alla semplificazione di molte attività svolte dai dipendenti mediante moderni sistemi
	informatici e allo sviluppo di diversi servizi web ed applicazioni mobile con l’obiettivo di fornire più informazioni possibili (inclusi foto e video) ai clienti.
	
	
	\begin{figure}[H]
		\centering
		\includegraphics[width=0.5\linewidth]{immagini/logo-vic}
		\caption{Il logo di VIC}
		\label{fig:logo-vic}
	\end{figure}
	 

\section{Il Dipartimento di Ricerca e Sviluppo} % DONE
	Il Dipartimento di Ricerca e Sviluppo (abbreviato: R\&D) ha sede a Padova e si occupa di realizzare e manutenere tutti gli strumenti che l'azienda offre, ad esempio:
	\begin{itemize}
		\item il portale \emph{VIC Online} per i clienti, mediante cui è possibile controllare lo stato dei propri ordini, ottenere informazioni utili, immagini e video riguardanti la merce ordinata;
		\item le applicazioni rivolte agli ispettori per raccogliere i dati sulla merce e realizzare dei resoconti sulle analisi effettuate e lo stato della merce.
	\end{itemize}
	 
	Il dipartimento R\&D é alla continua ricerca di metodi di innovazione per velocizzare, automatizzare e standardizzare le attività svolte dall’azienda nell’ambito del controllo qualità e quantità delle merci.
	
	Tutti i servizi sviluppati dal dipartimento R\&D sono
	composti dalle componenti front-end (client) e back-end (server). La comunicazione tra
	queste due componenti avviene utilizzando le quattro operazioni \gls{crud} di \gls{http}, in modo da essere conformi ai principi \gls{restg}.

\section{Opportunità di Stage} % DONE
	\vic{} è favorevole ad ospitare laureandi per lo stage perché crede che questa attività porti molteplici benefici all’azienda, nello specifico: \\
	\begin{itemize}
		\item il confronto con persone nuove e la loro esperienza può offrire spunti di miglioramento per i sistemi e le metodologie;
		\item per lo studio di nuove tecnologie e sul loro possibile utilizzo in progetti aziendali;
		\item la conoscenza di possibili candidati per l’assunzione.
	\end{itemize}
