% !TEX encoding = UTF-8
% !TEX TS-program = pdflatex
% !TEX root = ../tesi.tex
% !TEX spellcheck = it-IT

%**************************************************************
\chapter{L'Azienda}
\label{cap:azienda}

\intro{Questo capitolo descrive la realtà aziendale e il suo ambito di interesse}

%**************************************************************

\section{Chi è VIC} % DONE
	\paragraph{} 
	\vic{}\cite{site:vic} offre servizi di ispezione delle merci in tutto il mondo per individuare eventuali non conformità delle merci al fine di tutelare i propri i clienti e ridurre al minimo le perdite.
	
	VIC nasce a Venezia nel 2006 ed ora è una delle più grandi compagnie del mercato globale in quanto 
	è presente con 22 diverse aziende e più di 50 laboratori in tutti i più grandi snodi di scambio del mondo.
	
	\paragraph{} La chiave del successo di \vic{} è l'attenzione che pone alla riduzione del tempo tra ispezione e reporting al cliente. 
	Questo è stato possibile grazie all'automazione e semplificazione di molte attività svolte dai dipendenti mediante moderni sistemi
	informatici e allo sviluppo di diversi servizi web ed applicazioni mobile con l’obiettivo di fornire più informazioni possibili (inclusi foto e video) ai clienti.
	
	
	\begin{figure}[ht]
		\centering
		\includegraphics[width=0.5\linewidth]{immagini/logo-vic}
		\caption{Il logo di VIC}
		\label{fig:logo-vic}
	\end{figure}
	 

\section{Il Dipartimento R\&D}
	Tutti gli strumenti che  \vic{} offre ai dipendenti e ai clienti sono realizzati dal proprio Dipartimento di Ricerca e Sviluppo (abbreviato: R\&D).
	
	Esso ha sede a Padova e si occupa
	
	
	imponendo un approccio standard nelle operazioni di ispezione.
	
	
	Il dipartimento IT ha sviluppato diversi strumenti per migliorare e standardizzare le
	attività svolte dall’azienda nell’ambito del controllo qualità e quantità delle merci.

		Tutto il software sviluppato nel dipartimento IT di Padova viene utilizzato in tutte le
	sedi della società nel mondo.
	
	Il dipartimento IT è alla continua ricerca di metodi di innovazione delle attività
	aziendali. Lo dimostrano le proposte di stage presentate dall’azienda durante STAGE
	IT. 
	
	
	Il dipartimento IT ha sviluppato diversi strumenti per migliorare e standardizzare le
	attività svolte dall’azienda nell’ambito del controllo qualità e quantità delle merci.
	L’ispettore ha a disposizione un’applicazione mobile per Android che gli consente di
	inserire tutte le informazioni riguardanti lo stato della merce, immagini e video. Dovrà
	inoltre compilare anche una lista di controllo creata appositamente in base al tipo di
	merce ispezionata, che contiene una serie di domande sull’andamento dell’ispezione,
	sulla condizione della merce e del mezzo di trasporto (solitamente una nave).
	Il personale dell’azienda e i clienti potranno visualizzare tutte le informazioni inserite
	dall’ispettore. I dipendenti d’ufficio potranno utilizzare il servizio web per la gestione
	degli ordini allo scopo di recuperare le informazioni quando richiesto, comunicare con il
	cliente e per generare il report dell’ispezione.
	I clienti potranno accedere al loro spazio privato sul portale web VIC online, che gli
	consentirà di ottenere informazioni riguardo lo stato dell’ordine, le immagini, i video e
	le statistiche sulle condizioni delle merci.
	
	Come le moderne applicazioni web, tutti i servizi sviluppati dal dipartimento IT sono
	composti dalle componenti front-end (client) e back-end (server ). La comunicazione tra
	queste due componenti avviene utilizzando le operazioni CRUD (Create, read, update
	and delete) di HTTP, in modo da essere conformi ai principi REST (REpresentational
	State Transfer).
	

\section{Opportunità di Stage} 
	\vic{} considera gli stage come un’attività che porta valore aggiunto all’azienda. \\
	L’azienda vede nello stage principalmente due grandi vantaggi,
	momenti di conoscenza e confronto con l’esterno:
	\begin{itemize}
		\item ospitare studenti universitari comporta la possibilità di avere un confronto con
			le competenze e le conoscenze erogate dalle università agli studenti;
		\item l’azienda è sempre propensa a conoscere persone nuove e ad accogliere positivamente
			le loro proposte circa metodologie e tecnologie da adottare. Potrebbe
			infatti trovare, attivando stage, occasione di rinnovarsi e di introdurre metodologie
			più adatte al lavoro che svolge, oppure, scoprire e adottare nuove tecnologie
			che meglio si adattano ai suoi bisogni.
		\item dipartimento sfrutta gli stage anche per sondare, in caso di necessità, dei possibili
		candidati per l’assunzione.
	\end{itemize}
