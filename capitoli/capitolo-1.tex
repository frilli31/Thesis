% !TEX encoding = UTF-8
% !TEX TS-program = pdflatex
% !TEX root = ../tesi.tex

%**************************************************************
\chapter{L'Azienda}
\label{cap:azienda}

\intro{Descrizione dell'azienza, ambito di interesse, struttura e interesse nell'offrire opportunità di stage }
%**************************************************************

\section{Chi è VIC}
	Ogni minuto, in tutto il mondo avvengono accordi commerciali che coinvolgono la
	movimentazione di beni. VIC offre servizi di ispezione delle merci in 
	tutto il mondo per individuare eventuali non conformità delle merci 
	al fine di tutelare i propri i clienti e ridurre al minimo le perdite.
	
	VIC nasce a Venezia nel 2006 ed ora è una delle più grandi compagnie del mercato globale in quanto 
	è presente con 22 diverse aziende e più di 50 laboratori in tutti i più grandi snodi di scambio del mondo.
	
	La chiave del successo di VIC è l'attenzione che pone alla riduzione del tempo tra ispezione e reporting al cliente. 
	% TODO
	Indispensabile a questo scopo è il sistema telematico all'avanguardia, grazie massicci investimenti tecnologici che VIC continua a fare.
	
	
	\begin{figure}[ht]
		\centering
		\includegraphics[width=0.5\linewidth]{immagini/logo-vic}
		\caption{Il logo di VIC}
		\label{fig:logo-vic}
	\end{figure}


\section{Prodotti e servizi} % TODO
	Attraverso il portale VIC Online, i clienti possono accedere alla loro sezione privata da ogni parte del mondo e 
	verificare lo stato dei loro ordini come informazioni, immagini e video relativi allo spostamento delle loro merci.
	In questo modo il proprietario della merce può verificare le condizioni dei suoi prodotti in qualsiasi momento.
	Ecco le app specifiche per il tipo di ispezione da eseguire:
	\begin{itemize}
		\item Vic Damage Control (VDC): questo sistema aiuta l'operatore a controllare e segnalare i danni alla merce, a scattare foto dei danni e a segnalare i dettagli al cliente in tempo reale.
		\item Vic Cargo Control (VCC): questo sistema consente all'operatore di eseguire ispezioni visive delle merci attraverso una lista di controllo sincronizzata con il server
		\item Vic Weight Control (VWC): questo sistema aiuta l'operatore ad eseguire controlli del peso sulle merci scaricate.
		\item Vic Watch Online (VWO): questo sistema consente all'operatore di riprendere video delle operazioni di carico / scarico. 
			In questo modo il proprietario della merce può visualizzare il video e notare se qualcosa sta andando storto come se fosse lì.
	\end{itemize}
	 

\section{Il Dipartimento IT} % TODO
	Il Dipartimento IT è composto da personale laureato che ha un passato nella comunità del software libero, sviluppato per alcuni dei più 
	importanti progetti open source come OpenBSD ed il kernel Linux ed anche per progetti commerciali in una varietà di settori produttivi
	L'obiettivo del Dipartimento IT è quello di portare VIC nel futuro.
	Infatti, il dipartimento ha sviluppato un'infrastruttura rivoluzionaria che ha modernizzato il controllo qualitativo e quantitativo delle merci introducendo 
	un fondamentale aiuto tecnologico e imponendo un approccio standard nelle operazioni di ispezione.
	L'infrastruttura IT è composta da un gruppo centrale di server basati su GNU / Linux ed una serie di applicazioni mobile che possono essere sincronizzate in 
	qualsiasi luogo, al fine di eseguire l'ispezione. In particolare, le applicazioni sono utilizzate per raccogliere dati, 
	immagini e video a della movimentazione della merce in tempo reale.
	Queste app vengono usate dai dipendenti e dai fornitori in tutto il mondo. I dati vengono elaborati nei server centrali e resi accessibili al personale e ai clienti della VIC.
	


\section{Ricerca e Innovazione} % TODO Speggiorin Federica
	Il dipartimento IT ha sviluppato diversi strumenti per migliorare e standardizzare le
	attività svolte dall’azienda nell’ambito del controllo qualità e quantità delle merci.
	L’ispettore ha a disposizione un’applicazione mobile per Android che gli consente di
	inserire tutte le informazioni riguardanti lo stato della merce, immagini e video. Dovrà
	inoltre compilare anche una lista di controllo creata appositamente in base al tipo di
	merce ispezionata, che contiene una serie di domande sull’andamento dell’ispezione,
	sulla condizione della merce e del mezzo di trasporto (solitamente una nave).
	Il personale dell’azienda e i clienti potranno visualizzare tutte le informazioni inserite
	dall’ispettore. I dipendenti d’ufficio potranno utilizzare il servizio web per la gestione
	degli ordini allo scopo di recuperare le informazioni quando richiesto, comunicare con il
	cliente e per generare il report dell’ispezione. Il report viene compilato per la maggior
	parte automaticamente grazie alle informazioni contenute nel sistema, c’è quindi a
	tutti gli effetti un modello standard per i documenti.
	I clienti potranno accedere al loro spazio privato sul portale web VIC online, che gli
	consentirà di ottenere informazioni riguardo lo stato dell’ordine, le immagini, i video e
	le statistiche sulle condizioni delle merci.
	Tutte le informazioni vengono salvate su un unico server centralizzato, il quale permette
	a tutte le applicazioni sviluppate dal dipartimento IT di rimanere sincronizzate,
	evitando quindi inconsistenze tra i dati.
	Come le moderne applicazioni web, tutti i servizi sviluppati dal dipartimento IT sono
	composti dalle componenti front-end (client) e back-end (server ). La comunicazione tra
	queste due componenti avviene utilizzando le operazioni CRUD (Create, read, update
	and delete) di HTTP, in modo da essere conformi ai principi REST (REpresentational
	State Transfer).
	Tutto il software sviluppato nel dipartimento IT di Padova viene utilizzato in tutte le
	sedi della società nel mondo.
	Il dipartimento IT è alla continua ricerca di metodi di innovazione delle attività
	aziendali. Lo dimostrano le proposte di stage presentate dall’azienda durante STAGE
	IT. Per esempio, l’adozione del tablet Tango di Google, il quale fornisce una serie
	di sensori in grado di analizzare l’ambiente circostante, che l’azienda ha cercato di
	sfruttare per migliorare l’attività ispettiva della merce, cercando di rilevare eventuali
	danneggiamenti non visibili all’occhio umano o comunque inavvertitamente non visti
	dall’ispettore. Un altro esempio più recente è la sperimentazione dell’utilizzo di droni
	per filmare la merce nel caso si trovi in luoghi non raggiungibili dall’ispettore, per poi
	rendere disponibili i video al cliente.
 	% by Navid  

\section{Opportunità di Stage} % TODO by speggiorin federica, bottaro, navid 
	VIC considera gli stage come un’attività che porta valore aggiunto all’azienda.
	Non si tratta, infatti, di ospitare uno studente per due mesi al puro scopo di disporre di
	maggiore forza lavoro. L’azienda vede nello stage principalmente due grandi vantaggi,
	momenti di conoscenza e confronto con l’esterno:
	\begin{itemize}
		\item ospitare studenti universitari comporta la possibilità di avere un confronto con
			le competenze e le conoscenze erogate dalle università agli studenti;
		\item l’azienda è sempre propensa a conoscere persone nuove e ad accogliere positivamente
			le loro proposte circa metodologie e tecnologie da adottare. Potrebbe
			infatti trovare, attivando stage, occasione di rinnovarsi e di introdurre metodologie
			più adatte al lavoro che svolge, oppure, scoprire e adottare nuove tecnologie
			che meglio si adattano ai suoi bisogni.
	\end{itemize}
	
	Il dipartimento sfrutta gli stage anche per sondare, in caso di necessità, dei possibili
	candidati per l’assunzione. Allo scopo di promuovere le proprie offerte di stage, l’azienda
	partecipa agli eventi STAGE IT, in cui può contare sulla presenza di un ampio numero
	di studenti proveniente da diverse lauree con componente nell’ambito dell’Information
	and Communication Technology (ICT).