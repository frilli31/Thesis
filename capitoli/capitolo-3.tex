% !TEX encoding = UTF-8
% !TEX TS-program = pdflatex
% !TEX root = ../tesi.tex

%**************************************************************
\chapter{Svolgimento}
\label{cap:svolgimento}
%**************************************************************

\intro{Il presente capitolo descrive le varie fasi dello sviluppo dell'applicazione}

\section{Pianificazione}
	Come modello per la gestione del progetto ho adottato la metodologia \gls{agileg}. Tale
	modello mi ha permesso di adottare una certa flessibilità e velocità nel rispondere alle
	esigenze e ai feedback del Dipartimento di Amministrazione.
	
	\begin{figure}[ht]
		\centering
		\includegraphics[width=0.7\linewidth]{immagini/agile}
		\caption{Metologia Agile \newline  Fonte: \url{https://www.cleart.com/what-is-agile-software-development.html}}
		\label{fig:agile}
	\end{figure}
	
	Il modello Agile prevede iterazioni continue, della durata massima di due settimane,
	entro le quali si susseguono attività di:
	\begin{enumerate}
		\item analisi dei requisiti che emergono dall’interazione con il cliente, in questo caso il tutor aziendale e il Dipartimento di Amministrazione;
		\item pianificazione delle funzionalità da includere nello sprint, suddivisione in task e loro assegnazione ai membri del team;
		\item progettazione delle funzionalità da implementare nell’iterazione corrente;
		\item codifica e sviluppo delle funzionalità previste;
		\item verifica;
		\item rilascio;
		\item monitoraggio continuo e verifica dello stato dell’applicazione.
	\end{enumerate}
		
	Il punto di partenza di ogni ciclo è il risultato raggiunto con il precedente e il feedback
	ricevuto tramite la presentazione della soluzione raggiunta fin’ora agli stakeholder.
	
	La duranta dello stage è stata di 8 settimane quindi ho pianificato 4 sprint della durata di due settimane:
	\begin{enumerate}
		\item studio delle tecnologie, degli strumenti di sviluppo e del sotware gestionale esistente;
		\item progettazione, codifica e testing del modulo Banche;
		\item progettazione, codifica e testing del modulo Fatture Attive;
		\item progettazione, codifica e testing del modulo Fatture Passive.
	\end{enumerate}
			

\section{Tecnologie adottate}

	\subsection{Django}
		Per creare il back-end delle applicazioni rivolte ai clienti, il dipartimento utilizza Django,
		un framework di alto livello gratuito e open source scritto in Python, che incoraggia lo
		sviluppo veloce e pulito di applicazioni web.
	

\section{Progettazione}
	L’architettura del prodotto si suddivide in due macro componenti:
	\begin{itemize}
		\item \textbf{Client}: rappresenta l’applicativo desktop attraverso cui gli utenti interagis-
		cono con il front-end dell’applicazione. Il progetto si focalizza sullo sviluppo
		di questa componente.
		\item \textbf{Server}: rappresenta il sistema remoto che si occupa di calcolare e rispondere
		alle richieste del Client, viene implementato tramite il back-end. Lo sviluppo
		di questa componente non è pertinente al progetto ed è assegnata ad altri
		dipendenti dell’azienda.
	\end{itemize}

	Il Client interagirà con il Server tramite delle API messe pubblicate, che sono
	conformi ai principi REST.
	
	Tecnologie adottate
	
	Strumenti

\section{Realizzazione}

\section{Verifica e Validazione} % by Navid
	Durante l’intera attività di sviluppo ho verificato che quanto svolto fosse conforme alle
	mie aspettative e a quelle del cliente. Per far ciò ho attuato molte prove pratiche sulle
	funzionalità implementate, per assicurarmi che funzionassero in modo adeguato.
	L’azienda ha preferito utilizzare le prove pratiche piuttosto che quelle automatizzate,
	in modo che potessi concentrarmi sull’adeguare il prodotto secondo le loro esigenze.
	Grazie al lavoro svolto in fase di configurazione dell’ambiente di lavoro, ho potuto
	testare velocemente l’applicazione in locale con la base di dati contenente i dati reali.
	
	\subsection{Analisi Statica} 
	Perl’attività di verifica durante tutto l’arco di sviluppo del progetto ho utilizzato TSHint
	tramite il pacchetto Atom-jshint creato appositamente per l’editor Atom. JSHint è uno
	strumento di analisi statica del codice JavaScript sviluppato a partire da un fork di
	JSLint, in quanto l’originale risultava essere troppo meticoloso e non personalizzabile.
	JSHint mi ha permesso di controllare che il codice fosse uniforme e scritto secondo dei
	buoni canoni, grazie a degli appositi warning ed errori.
	% scrennshot di PyCharm e WebStrorm
	
	\subsection{Analisi dinamica}
	Per l’analisi del prodotto software tramite la sua esecuzione mi sono avvalso di Geny-
	motion, un’applicazione che consente l’emulazione di dispositivi mobile. Ho dapprima
	richiesto al tutor aziendale quali fossero le configurazioni tipo su cui l’applicazione
	sarebbe stata eseguita, così da impostare le macchine virtuali su Genymotion. Successivamente, mi sono assicurato di eseguire dei test mirati a controllare il funzionamento
	delle specifiche funzionalità dell’applicazione, utilizzando i dati reali contenuti nella
	base di dati copia fornitami dall’azienda.
	
	Ho individuato le seguenti funzionalità principali che nel complesso formano l’ap-
	plicazione:
	......

\section{Risultato finale}
	Descrizione dell'applicazione con screenshot
	\subsection{Modifiche Migliorative}
