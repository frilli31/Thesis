% !TEX encoding = UTF-8
% !TEX TS-program = pdflatex
% !TEX root = ../tesi.tex

%**************************************************************

\chapter{API Rest}

\section{Introduzione}
La prima definizione di \gls{apig} \gls{restg} si ha grazie alla tesi
"\textit{Architectural Styles and the Design of Network-based Software
	Architectures}" dello studente \textbf{Roy Fielding} negli anni 2000, che
espone un nuovo tipo di architettura per sistemi di ipertesto distribuiti, come
ad esempio il \textit{World Wide Web}.

\section{REST}

\gls{restg} detta un insieme di linee guida riassumibili in cinque principi:
\begin{enumerate}
	\item identificazione delle risorse;
	\item utilizzo esplicito di metodi \gls{http};
	\item risorse autodescrittive;
	\item collegamenti tra risorse;
	\item comunicazioni prive di stato.
\end{enumerate}

\subsection{Identificazione delle risorse}
Per risorsa si intende un qualsiasi elemento oggetto di elaborazione.
Le risorse rappresentano un concetto fondamentale nelle \gls{apig} di tipo
\gls{restg}. Queste risorse vengono richieste tipicamente da una macchia
denominata \textit{client} verso una macchina denominata \textit{server}. Le
richieste vengono solitamente effettuate tramite il protocollo \gls{http}, ma è
possibile utilizzare un tipo di protocollo diverso. È possibile stabilire anche
il formato delle risorse.

\subsection{Utilizzo esplicito di metodi \gls{http}}
Le richieste se effettuate tramite l'utilizzo del protocollo \gls{http} possono
essere eseguite utilizzando i metodi CRUD. Il protocollo \gls{http}
mappa i metodi \gls{crud} nella seguente maniera:
\begin{itemize}
	\item \textbf{create} è associato a richieste di tipo PUT o POST;
	\item \textbf{read} è associato a richieste di tipo GET;
	\item \textbf{update} è associato a richieste di tipo PUT o POST;
	\item \textbf{delete} è associato a richieste di tipo DELETE.
\end{itemize}

\subsection{Risorse autodescrittive}
I principi \gls{restg} non impongono vincoli sulle modalità di rappresentazione
di una risorsa. L'utilizzo di formati standard però è consigliato, in maniera
tale da avere una migliore comunicazione con i client. Per ottenere una
rappresentazione standard è possibile rappresentare le risorse con l'adeguato
tipo mime, per contro, un client ha la possibilità di richiedere
la risorsa secondo uno specifico formato: in questa maniera, il server
può fornire più modi di rappresentare per una risorsa.

\subsection{Collegamenti tra risorse}
È possibile che siano presenti relazioni tra le risorse. Un metodo consigliato
per rappresentarle e distinguerle è tramite l'utilizzo di un identificatore
ogni volta diverso.
In questa maniera, è possibile avere diversi collegamenti tra risorse tramite
un \gls{url} univoco.

\subsection{Comunicazioni prive di stato}
Questa caratteristica è possibile trovarla proprio nel protocollo \gls{http},
dove non si mantiene un collegamento attivo, non esiste il concetto di
"sessione" e ciascuna richiesta non ha alcuna relazione con le richieste
precedenti o successive. Questo concetto è stato portato anche nei servizi di
tipo \gls{restg}.
In questa maniera è il client che si deve occupare di mantenere lo
stato dell'applicazione, permettendo di avere un'alta scalabilità lato
\textit{server}, in quanto non sono presenti problemi di sincronizzazione delle
varie sessioni o mantenimenti di stato tra le varie macchine.



