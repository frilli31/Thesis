\section{Analisi dei requisiti}
%**************************************************************
Nella fase di analisi dei requisiti, ho avuto modo di discutere a fondo con il tutor
aziendale le funzionalità che il prodotto doveva rendere disponibili, aiutandoci anche con lo studio delle funzionalità disponibili nel vecchio gestionale ancora operativo.

Nella tabelle che seguono verranno presentati i principali requisiti individuati
durante l’analisi del problema e della soluzione già implementata nel vecchio applicativo, discussi con il tutor e il Dipartimento di Amministrazione.
Ogni requisito individuato avrà un codice identificativo univoco così formato:
Il codice dei requisiti è così strutturato R(F/Q/V)(N/D/O) dove:
\begin{enumerate}
	\item[R =] requisito;
    \item[F =] funzionale;
    \item[Q =] qualitativo;
    \item[V =] di vincolo;
    \item[N =] obbligatorio (necessario);
    \item[D =] desiderabile;
    \item[Z =] opzionale.
\end{enumerate}

\newpage

\begin{table}%
\caption{Tabella del tracciamento dei requisti funzionali}
\label{tab:requisiti-funzionali}
\begin{tabularx}{\textwidth}{lXl}
\hline\hline
\textbf{Requisito} & \textbf{Descrizione} & \textbf{Importanza}\\
\hline
RFN-1     & L'interfaccia permette di configurare il tipo di sonde del test & Obbligatorio \\
\hline
\end{tabularx}
\end{table}%

\begin{table}%
\caption{Tabella del tracciamento dei requisiti qualitativi}
\label{tab:requisiti-qualitativi}
\begin{tabularx}{\textwidth}{lXl}
\hline\hline
\textbf{Requisito} & \textbf{Descrizione} & \textbf{Importanza}\\
\hline
RQD-1    & Le prestazioni del simulatore hardware deve garantire la giusta esecuzione dei test e non la generazione di falsi negativi & Obbligatorio \\
\hline
\end{tabularx}
\end{table}%

\begin{table}%
\caption{Tabella del tracciamento dei requisiti di vincolo}
\label{tab:requisiti-vincolo}
\begin{tabularx}{\textwidth}{lXl}
\hline\hline
\textbf{Requisito} & \textbf{Descrizione} & \textbf{Importanza}\\
\hline
RVO-1    & La libreria per l'esecuzione dei test automatici deve essere riutilizzabile & - \\
\hline
\end{tabularx}
\end{table}%