% !TEX encoding = UTF-8
% !TEX TS-program = pdflatex
% !TEX root = ../tesi.tex

%**************************************************************
\chapter{Il Progetto}
\label{cap:progetto}

\intro{Descrizione del progetto di stage, motivazioni, obbiettivi.}
%**************************************************************

\section{L'applicativo}

	L’azienda utilizza un software gestionale ricco di funzionalità, ma molto datato.
	Esso infatti è stato sviluppato interamente in \gls{phpg}\cite{site:php} ed è basato sul framework \textit{QCubed}\cite{site:QCubed} che
	sta per concludere il suo ciclo di vita in quanto non è supportato più supportato e non è compatibile con le nuove versioni di \gls{phpg}.
	\paragraph{\\ \\}
	Lo scopo dello stage è stato quello di sviluppare un nuovo gestionale che si interfacci con lo stesso database del precedente e con le stesse
	funzionalità, ma basato su tecnologie moderne, prestando cura anche al rinnovamento
	dell’interfaccia grafica.
	
	Nello specifico il software da realizzare rappresenta il Modulo Amministrativo: esso è uno dei moduli principali in cui è diviso il
	software di gestione in uso attualmente presso l’azienda ed si occupa di gestire tutte le fatture e tenere sotto controllo la gestione contabile.
	
	Esso si divide in tre sottomoduli: Banche, Fatture Attive e Fatture Passive.
	
	\subsection{Banche}
	Il modulo banche si occupa di tenere sotto controllo la situazione contabile dei vari conti correnti aziendali.\\
	Dovrà offrire le seguenti funzionalità:
	\begin{itemize}
		\item visualizzazione della lista delle banche associata alla sede corrente mostrandone per ognuna:
			\begin{itemize}
				\item il saldo reale (inserito manualmente dagli operatori) di ogni banca;
				\item la data di inserimento del saldo reale;
				\item il saldo calcolato dal sistema in base ai movimenti bancari inseriti (che dovrebbe coincidere con il saldo reale, se aggiornato);
				\item il fido di cassa, cioè la somma che la banca mette a disposizione dell' azienda;
				\item il saldo disponibile, ottenuto sommando il saldo reale al, se presente, il fido di cassa;
			\end{itemize}
		\item visione di tutti i movimenti di un conto corrente;
		\item aggiornamento del saldo di un conto corrente;
		\item caricamento e visione di estratti conti giornalieri e mensili;
		\item inserimento, modifica e rimozione di movimenti bancari;
		\item generazione di file nei formati \textit{PDF} e \textit{xls} della lista dei movimenti bancari.
	\end{itemize}
	
	\subsection{Fatture attive}
	Il modulo delle fatture attive si occupa di tenere traccia di tutti gli ordini completati e in attesa
	di fatturazione, della creazione della fattura stessa con la possibilità di specificare i parametri
	richiesti (data fattura, scadenza, tipo fattura…), della possibilità di inviare le fatture al cliente e
	la gestione dello scadenziario delle fatture scadute e non ancora incassate.\\
	Dovrà offrire le seguenti funzionalità:
	\begin{itemize}
		\item visualizzazione degli ordini completati e non ancora fatturati di cui è possibile creare la fattura;
		\item ricerca di fatture attraverso il numero di ordine o il nome della nave;
		\item creazione, modifica e rimozione di una fattura;
		\item visualizzazione della lista di fatture non ancora inviate al cliente;
		\item generazione di file in formato \textit{PDF} corrispondenti alle fatture;
		\item invio di fatture al cliente tramite email;
		\item scadenziario fatture emesse da incassare
		\begin{itemize}
			\item già scadute;
			\item in scadenza.
		\end{itemize}
	\end{itemize}
	
	\subsection{Fatture passive}
	Il modulo delle fatture passive si occupa di tenere traccia di tutte le passività relative
	all’attività. Si dividono in due gruppi: dirette e indirette. Le fatture passive dirette rappresentano
	delle passività direttamente riconducibile ad un preciso ordine (fornitori, agenzie, laboratori,
	spedizionieri), mentre le indirette rappresentano dei costi generici di gestione (elettricità,
	bollette telefoniche, affitti,…). Le fatture passive indirette si suddividono a propria volta in
	due categorie: una tantum e cicliche. \\
	Dovrà offrire le seguenti funzionalità:
	\begin{itemize}
		\item visualizzazione lista passive dirette non pagate raggruppate per fornitore;
		\item visualizzazione lista passive indirette (divise tra cicliche e una tantum);
		\item visualizzazione lista fatture per ogni fornitore;
		\item visione dettaglio singola fattura;
		\item modifica fattura;
		\item eliminazione fattura;
		\item generazione del movimento bancario associato (con riempimento automatico dei campi).
	\end{itemize}	

\section{Obbiettivi}
Sulla base della durata massima di 320 ore prevista per lo stage, il tutor e lo stagista hanno
stilato un piano di lavoro e coerentemente hanno concordato gli obbiettivi minimi e massimi che si aspetta di veder raggiunti al termine del
rapporto lavorativo.
\\ \\
Gli obiettivi minimi concordati sono:
\begin{enumerate}
	\item comprensione del sotware e del database esistenti;
	\item studio delle tecnologie;
	\item studio degli strumenti di sviluppo;
	\item definire, progettare, codificare e verificare le funzionalità che riguardano i moduli:
		\begin{itemize}
			\item Banche;
			\item Fatture Attive.
		\end{itemize}
\end{enumerate}

Gli obiettivi massimi concordati sono:
\begin{enumerate}
	\item definire, progettare,  codificare e verificare le funzionalità che riguardano il modulo:
	\begin{itemize}
		\item Fatture Passive.
	\end{itemize}
\end{enumerate}
