% !TEX encoding = UTF-8
% !TEX TS-program = pdflatex
% !TEX root = ../tesi.tex

%**************************************************************
\chapter{Il Progetto}
\label{cap:progetto}

\intro{Descrizione del progetto di stage, motivazioni, tecnologie interessate, obbiettivi}
%**************************************************************

\section{L'applicativo}
	L’azienda utilizza un software gestionale ricco di funzionalità, ma molto datato.
	Esso infatti è stato sviluppato interamente in PHP ed è basato sul framework QCubed\cite{site:QCubed} che
	sta per concludere il suo ciclo di vita in quanto è supportato solo da PHP 5.6, una 
	versione ormai sorpassata da PHP 7 che lo ha sostituito come pacchetto predefinito nel sistema operativo Ubuntu
	a partire dalla versione 16.04, che vincola il mantenimento del sistema operativo 14.04.
	
	Lo scopo dello stage è stato quello di sviluppare un nuovo gestionale, con le stesse
	funzionalità, ma basato su tecnologie moderne, prestando cura anche al rinnovamento
	dell’interfaccia grafica. 
	
	Nello specifico il software da realizzare  rappresenta il Modulo Amministrativo: esso è uno dei moduli principali in cui è diviso il
	software di gestione in uso attualmente presso l’azienda. Entra in gioco subito dopo il modulo operativo
	e si occupa di gestire tutti gli aspetti contabili degli ordini, una volta che questi sono portati a compimento
	dallo staff. Si divide a propria volta in tre sottomoduli riguardanti le fatture attive, le fatture passive e la
	gestione dei conti bancari.
	
	\subsection{Fatture attive}
	Il modulo delle fatture attive si occupa di tenere traccia di tutti gli ordini completati e in attesa
	di fatturazione, della creazione della fattura stessa con la possibilità di specificare i parametri
	richiesti (data fattura, scadenza, tipo fattura…), della possibilità di inviare le fatture al cliente e
	la gestione dello scadenziario delle fatture scadute e non ancora incassate.\\
	Dovrà offrire le seguenti funzionalità:
	\begin{itemize}
		\item R1: lista fatture da creare;
		\item R2: creazione nuova fattura;
		\item R3: modifica fattura esistente non ancora incassata;
		\item R4: rimozione fattura esistente non ancora incassata;
		\item R5: lista fatture create non ancora inviate al cliente;
		\item R6: invio fattura al cliente tramite email;
		\item R7: scadenziario fatture emesse da incassare
		\begin{itemize}
			\item R7.1: già scadute;
			\item R7.2: in scadenza.
		\end{itemize}
	\end{itemize}
	
	
	\subsection{Fatture passive}
	Il modulo delle fatture passive si occupa di tenere traccia di tutte le passività relative
	all’attività. Si dividono in gruppi: dirette e indirette. Le fatture passive dirette rappresentano
	delle passività direttamente riconducibile ad un preciso ordine (fornitori, agenzie, laboratori,
	spedizionieri), mentre le indirette rappresentano dei costi generici di gestione (elettricità,
	bollette telefoniche, affitti,…). Le fatture passive indirette si suddividono a propria volta in
	due categorie: una tantum e cicliche. \\
	Dovrà offrire le seguenti funzionalità:
	\begin{itemize}
		\item R1: Lista passive dirette non pagate raggruppate per fornitore
		\begin{itemize}
			\item R1.1: visione lista fatture per ogni fornitore
			\item R1.2: visione dettaglio singola fattura
			\item R1.3: modifica fattura
			\item R1.4: eliminazione fattura
		\end{itemize}
		\item R2: Lista passive indirette (divise tra cicliche e una tantum)
		\begin{itemize}
			\item R2.1: visione lista fatture per ogni fornitore
			\item R2.2: visione dettaglio singola fattura
			\item R2.3: modifica fattura
			\item R2.4: eliminazione fattura
		\end{itemize}
	\end{itemize}
	
	\subsection{Banche}
	Il modulo banche si occupa di tenere sotto controllo la situazione contabile dei vari conti
	correnti aziendali.\\
	Dovrà offrire le seguenti funzionalità:
	\begin{itemize}
		\item R1: Visione saldi conti correnti banche
		\item R2: Visualizzazione fido di cassa
		\item R3: Visualizzazione saldo reale e saldo disponibile
	\end{itemize}
		

\section{Tecnologie Adottate}


\section{Strumenti}


