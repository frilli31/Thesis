% !TEX encoding = UTF-8
% !TEX TS-program = pdflatex
% !TEX root = ../tesi.tex

%**************************************************************
\chapter{Analisi retrospettiva}
\label{cap:valutazione}
%**************************************************************

\intro{Considerazioni sull'esperienza maturata durante lo stage e, più in generale, attraverso l'intero
	percorso formativo.}

\section{Bilancio formativo}

Dal punto di vista formativo l’attività di stage è stata estremamente positiva. Ha
arricchito il mio bagaglio personale di competenze professionali non banali.

Nonostante lo stage in questione abbia richiesto una considerevole mole di studio ed
autoformazione, gli argomenti trattati e di ricerca sono risultati piacevoli, interessanti,
utili

mondo aziendale diverso da mondo universitario:
Esperienza di integrazione in una realtà aziendale con i vari aspetti positivi (colleghi, tutor che aiuta), ma anche negativi l'ambiente di lavoro non ideale secondo i principi dell'ingengeria del software (imprevisiti, urgenze, scadenze, non conoscenza del dominio in cui , il confronto con altre persone, la carenza di documentazione e best practice da seguire)

Durante tutto lo stage ho dovuto lavorare da sola, visto che il progetto era stato apposi-
tamente calibrato in questa modalit`a. Questo mi ha permesso di raggiungere un grado
di autonomia che fin’ora non possedevo.
Ho comunque avuto la possibilit`a di confrontarmi con personale qualificato in ogni
momento di difficolt`a. Solitamente quando sono bloccata su un problema, preferisco
cercare di risolverlo da sola, piuttosto che chiedere aiuto a qualcuno. Durante lo stage
ho appreso che questa mia metodologia non `e del tutto corretta: ho imparato a confrontarmi nei momenti di necessit`a per raggiungere in tempi pi`

Questo insegnamento mi sar`a sicuramente utile se in futuro lavorer`o in un team di
sviluppo.

\section{Competenze Tecnologiche}
Durante lo stage ho avuto modo di studiare in modo più approfondito le tecnologie e usarle in ambiente lavorativo; questo mi ha permesso di comprenderle meglio, capirne pregi e difetti e farne un utilizzo più consapevole.
Nello specifico ho migliorato le mie conoscenze riguardanti i seguenti argomenti:
	\begin{itemize}
		\item \textbf{la metodologia Agile}: di cui ho apprezzato la flessibilità e leggerezza;
		\item \textbf{lo stile architetturale REST};
		\item il linguaggio \textbf{Python}: un linguaggio molto intuitivo, semplice e conciso; per contro è un linguaggio a tipizzazione dinamica per cui è necessaria l'esecuzione del codice per rivelare eventuali e provocando ingenti perdite di tempo per debugging e testing. A causa questo difetto credo che lungo periodo e/o in progetti impegnativi sia più conveniente utilizzare altri linguaggi, ad esempio Java.
		\item Django: supporto a ottima documentazione, query in modo semplice, ottima pagina di debugging
		\item Typescript: facile da imparare visto che è un sovrainsieme del già conosciuto Javascript, 
		\item Angular: ordinato, comoda separazione tra visualizzazione e comportamento, e tra servizi e pagine.
		\item l'utilizzo di \textbf{IntelliJ}(e le sue derivate PyCharm e WebStorm): eccezionale, pintegrazione con vistual envirornment, permette configurazione del sistema di building in modo semplice
		per integrazione con Git, sistema di navigazione e ricerca, sistema di versionamento integrato (per ogni modifica), integrazione con accesso a database, Rest client, integra il controllo del codice e di sintassi
	\end{itemize}.
	

\section{Esperienza}
	La seconda parte che va a formare il bagaglio culturale creato con l’attività di stage è
	sicuramente l’esperienza aziendale, parte che tra le due ritengo la più importante.
	La collaborazione con i colleghi, il rapporto con il datore di lavoro e gli stakeholderG, il
	rispetto degli orari, della sicurezza e della regolamentazione costituiscono un’esperienza
	fondamentale per uno studente di un indirizzo orientato al mondo del lavoro. Poter
	usufruire di un servizio universitario, che garantisce un’esperienza lavorativa all’interno
	del percorso di studi, permette a tutti gli studenti che entrano per la prima volta nel
	mondo del lavoro di conoscere già le dinamiche generali aziendali così da poter fin da
	subito adattarvisi.
	Ritengo dunque il bilancio formativo davvero positivo e posso affermare che lo stage rappresenta una delle più importanti attività svolte nel mio personale percorso
	universitario.
	
	Consiglio pertanto l’attività dello stage anche a tutti quelli studenti di altri corsi di
	studi che non la prevedono come obbligatoria, in quanto solo tramite il suo svolgimento è possibile apprezzare nozioni che durante il normale corso di studi non è possibile
	apprendere.
