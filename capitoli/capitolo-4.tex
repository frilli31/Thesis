% !TEX encoding = UTF-8
% !TEX TS-program = pdflatex
% !TEX root = ../tesi.tex

%**************************************************************
\chapter{Analisi retrospettiva}
\label{cap:valutazione}
%**************************************************************

\intro{Considerazioni sulle conoscenze acquisite e sull'esperienza maturata durante lo stage}

\section{Competenze Acquisite}
Dal punto di vista formativo l’attività di stage è stata estremamente positiva e ha rispettato le mie aspettative iniziali. \\ \\
Ha arricchito il mio bagaglio personale di conscenze professionali importanti.
Le competenze che posso dire di aver aggiunto o ampliato rispetto al bagaglio formativo
in mio possesso prima dell’inizio dello stage sono molteplici e possono essere riassunte nei
seguenti aspetti:
\begin{itemize}
	\item \hyperref[sub:esplavorativa]{esperienza lavorativa};
	\item \hyperref[sub:tecnologie]{tecnologie};
	\item \hyperref[sub:lavoro]{lavoro individuale}.
\end{itemize}
	
	\subsection{Esperienza Lavorativa}
	\label{sub:esplavorativa}
	Non avendo esperienze lavorative, visto che in questi tre anni ho preferito concentrami
sugli studi, credo sia questo sia l'aspetto in cui io mi sia maggiormete arricchito. \\ \\
	Questo ambito comprende:
	\begin{itemize}
		\item integrarmi in una nuova realtà aziendale e l'apprendere le sue metodologie;
		\item la collaborazione e il confronto con i colleghi;
		\item il rapporto con il datore di lavoro e con gli stakeholder, nel mio caso il tutor e i membri del Dipartimento di Amministrazione;
		\item il rispetto degli orari e delle scadenze.
	\end{itemize}
	
	Durante questa esperienza ho affrontato alcune dificoltà e capito l'impatto che esse hanno, tra cui:
	\begin{itemize}
		\item la volatilità dei requisiti;
		\item l'assenza di linee guida scritte e procedure a livello aziendale per la progettazione e la scrittura del codice;
		\item la non conoscenza del dominio dell'applicativo, nello specifico l'ambito amministrativo e riguardante la fatturazione;
		\item la carenza di documentazione e la conseguente perdita di tempo nello studiare il codice prodotto da terzi.
	\end{itemize}
	
	\subsection{Tecnologie}
	\label{sub:tecnologie}
	Nonostante lo stage in questione abbia richiesto una considerevole mole di studio ed
	autoformazione, gli argomenti trattati e di ricerca sono risultati piacevoli, interessanti e
	utili. \\
	Durante lo stage ho avuto modo di studiare in modo approfondito alcune tecnologie e di usarle in ambiente lavorativo permettendomi di comprenderle meglio, capirne pregi e difetti e imparare a farne un utilizzo più consapevole. \\ \\
	Nello specifico ho migliorato le mie conoscenze riguardanti i seguenti argomenti:
		\begin{itemize}
			\item \textbf{la metodologia Agile}: di cui ho apprezzato la flessibilità e leggerezza, ma che ha creato difficoltà in alcuni momenti a causa dell'insufficienza di documentazione;
			\item \textbf{lo stile architetturale REST}: grazie alla sua semplicità e alla sua versatilita l'architettura con back-end, front-end e la comunicazione tramite \gls{apig} \gls{restg} viene utilizzata nella maggior parte delle applicazioni;
			\item il linguaggio \textbf{Python}: l'ho trovato molto intuitivo, semplice, conciso e molto ricco di librerie; per contro è un linguaggio a tipizzazione dinamica per cui è necessaria l'esecuzione del codice per rivelare la maggior parte degli errori provocando ingenti perdite di tempo per effettuare debugging e testing. A causa questo difetto credo che lungo periodo e in progetti di grandi dimensioni sia più conveniente utilizzare altri linguaggi, ad esempio Java;
			\item \textbf{Django}: l'ho trovato un ottimo framework in grado di rendere lo sviluppo di complesse applicazioni semplice e veloce;
			\item \textbf{Angular}: secondo la mia esperienza è ottimo per applicazioni complesse grazie all'architettura che impone, ma di difficile apprendimento e inutilmente prolisso quindi non adatto a sviluppatori poco esperti o progetti di piccole dimensioni;
			\item l'utilizzo di \textbf{IntelliJ}(e le sue derivate PyCharm e WebStorm): grazie all'elevato numero di strumenti integrati e alla possibilità di configurarlo secondo le proprie esigenze l'ho trovato estremante utile e il suo uso mi ha permesso di risparmiare molto tempo in sede di codifica e debug.
		\end{itemize}
			
	\label{sub:lavoro}
	\subsection{Lavoro individuale}
	Durante tutto lo stage ho dovuto lavorare da solo, visto che il progetto era stato appositamente calibrato in questa modalità. Questo mi ha permesso di raggiungere un grado di autonomia che fin’ora non possedevo.\\
	Ho comunque avuto la possibilità di confrontarmi con personale qualificato in ogni
	momento di difficoltà. Solitamente quando sono bloccato su un problema, preferisco
	cercare di risolverlo da solo, piuttosto che chiedere aiuto a qualcuno. Durante lo stage
	ho appreso che questa mia metodologia non è del tutto corretta: ho imparato a confrontarmi nei momenti di necessità per raggiungere in tempi più rapidi la soluzione.\\
	Questo insegnamento mi sarà sicuramente utile se in futuro lavorerò in un team di
	sviluppo.

\section{Valutazione Personale}
Nel complesso mi sento soddisfatto di questa esperienza che mi ha dato l’opportunità di entrare nel mondo
del lavoro e confrontarmi con esso costituendo un’importante opportunità di crescita
personale. \\
L'unico rammarico consiste nel fatto che la politica dell'azienda improntata principalmente al risultato non permetta di mettere molta attenzione alla qualità del software e alla documentazione, che sarebbe possibile attraverso l'investimento di più tempo nella formazione del personale e nell'adozione di linee guida e regole aziendali che normino tali attività.
\\ \\
Ritengo che l’università sia il luogo migliore per poter apprendere le basi di ciò che poi
si dovrà applicare alla vita lavorativa. \\
Poter usufruire di un servizio universitario, che garantisce un’esperienza lavorativa all’interno
del percorso di studi, permette a tutti gli studenti che entrano per la prima volta nel
mondo del lavoro di conoscere già le dinamiche generali aziendali così da poter fin da
subito adattarvisi. \\ \\
	Ritengo dunque il bilancio formativo davvero positivo e posso affermare che lo stage rappresenta una delle più importanti attività svolte nel mio personale percorso universitario. \\
Consiglio pertanto l’attività dello stage anche a tutti quelli studenti di altri corsi di
studi che non la prevedono come obbligatoria, in quanto solo tramite il suo svolgimento è possibile dare il giusto valore alle nozioni apprese durante il corso di studi.
