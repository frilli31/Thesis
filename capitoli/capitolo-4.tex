% !TEX encoding = UTF-8
% !TEX TS-program = pdflatex
% !TEX root = ../tesi.tex

%**************************************************************
\chapter{Analisi retrospettiva} % TODO : Valutazione, interpretazione dei risultati, considerazioni
\label{cap:valutazione}
%**************************************************************

\intro{Considerazioni sull'esperienza maturata durante lo stage e, più in generale, attraverso l'intero
	percorso formativo.}

mondo aziendale diverso da mondo universitario:
Esperienza di integrazione in una realtà aziendale con i vari aspetti positivi (colleghi, tutor che aiuta), ma anche negativi l'ambiente di lavoro non ideale secondo i principi dell'ingengeria del software (imprevisiti, urgenze, scadenze, non conoscenza del dominio in cui , il confronto con altre persone, la carenza di documentazione e best practice da seguire)

\section{Considerazioni sulle Tecnologie}
	Durante lo stage ho avuto modo di accrescere il mio bagaglio tecnologico approfondendo la conoscenza dei linguaggi Python e Typescipt e imparando come utilizzare i Framework Django e Angular.
	
	\subsection{Python e Django}
	Ho trovato Python un linguaggio molto intuitivo, semplice e veloce da imparare, conciso 
	dinamico, nessun controllo a compile time => necessaria l'esecuzione del codice, molto tempo perso per debug
	DJango: bello, query in modo semplice (problemi con fk) svantaggio di Python, 
	Concludendo nel lungo periodo credo sia più conveniente altre piattaforme, magari meno intuitive ma che supportino meglio il debugging.
	
	\subsection{Django}
	
	\subsection{Angular}
	ordinato, comoda separazione tra visualizzazione e comportamento, e tra servizi e pagine
	

\section{Considerazioni sugli Strumenti}
	L'IDE utilizzato è stato IntelliJ (e le sue derivate PyCharm e WebStorm)
	IntelliJ e derivati: eccezionale, pintegrazione con vistual envirornment, permette configurazione del sistema di building in modo semplice
	per integrazione con Git, sistema di navigazione e ricerca, sistema di versionamento integrato (per ogni modifica), integrazione con accesso a database, Rest client, integra il controllo del codice e di sintassi

\section{Esperienza}

\section{Valutazioni sul rapporto stage-laurea}
Analisi critica del rapporto formativo tra stage e corso di laurea

\section{Valuazioni sulla Laurea Triennale}
