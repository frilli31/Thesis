% !TEX encoding = UTF-8
% !TEX TS-program = pdflatex
% !TEX root = ../tesi.tex
% !TEX spellcheck = it-IT

%**************************************************************
% Sommario
%**************************************************************
\cleardoublepage
\phantomsection
\pdfbookmark{Sommario}{Sommario}
\begingroup
\let\clearpage\relax
\let\cleardoublepage\relax
\let\cleardoublepage\relax

\chapter*{Sommario}

Il presente documento descrive il lavoro svolto durante il periodo di stage dal laureando
Luca Allegro presso l'azienda \vic{} \\
Il torocinio si è tenuto presso il Dipartimento di Ricerca e Sviluppo (R\&D) durante i mesi di
luglio e agosto 2018 e ha avuto la durata di 310 ore.
\\ \\
L’obiettivo dello stage è stato lo sviluppo di un’applicazione web per la gestione amministrativa
dell'azienda che comprende il tenere traccia dei movimenti bancari e la gestione delle fatture attive e 
delle fatture passive.
%Al fine di poterla integrare nel sistema di gestione già presente in \vic{} è stato richiesto che
%fosse utili
% era richiesto di sviluppare l’applicazione tramite il framework scelto e un piccolo backend
%tramite l’utilizzo del framework Django con cui l’applicazione necessita di interagire.
\\ \\
L'elaborato ha lo scopo di illustrare:
\begin{itemize}
	\item il contesto aziendale dove è stato svolto lo stage ({\hyperref[cap:azienda]{Capitolo 1}});
	\item il progetto sviluppato ({\hyperref[cap:progetto]{Capitolo 2}});
	\item le attività svolte durante esso ({\hyperref[cap:svolgimento]{Capitolo 3}});
	\item valutazione finale sull'esperienza e le competenze acquisite({\hyperref[cap:valutazione]{Capitolo 4}})
\end{itemize}

\vspace{2cm}

\subsection*{Convenzioni tipografiche}
Nell’elaborato che segue sono state adottate alcune convenzioni tipografiche per
facilitarne la consultazione e la comprensione:
\begin{itemize}
	\item la versione digitale dell’elaborato conterrà collegamenti riconoscibili attraverso il
	colore azzuro;
	\item per non appesantire la lettura, il significato di alcuni termini sarà riportato nella
	sezione \hyperlink{Glossario}{Glossario}. Tutte le occorrenze di tali termini saranno in colore
	azzurro e se cliccati rimanderanno al sisgnificato;
	\item termini facenti parti del gergo tecnico sono evidenziati con il carattere corsivo;
	\item termini indicanti parti di codice sono evidenziati con il carattere monospace;
\end{itemize}
%\vfill
%
%\selectlanguage{english}
%\pdfbookmark{Abstract}{Abstract}
%\chapter*{Abstract}
%
%\selectlanguage{italian}

\endgroup			

\vfill

