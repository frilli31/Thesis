
%**************************************************************
% Acronimi
%**************************************************************
\renewcommand{\acronymname}{Acronimi e abbreviazioni}

\newacronym[description={\glslink{apig}{Application Program Interface}}]
    {api}{API}{Application Program Interface}

\newacronym[description={\glslink{restg}{REpresentational State Transfer}}]
    {rest}{REST}{REpresentational State Transfer}

\newacronym[description={\glslink{umlg}{Unified Modeling Language}}]
    {uml}{UML}{Unified Modeling Language}
    
\newacronym[description={\glslink{phpg}{PHP: Hypertext Preprocessor}}]
    {php}{PHP}{PHP: Hypertext Preprocessor}
    
\newacronym{http}{HTTP}{Hypertext Transfer Protocol}

\newacronym{crud}{CRUD}{Create Read Update Delete}

\newacronym{url}{URL}{Uniform Resource Locator}

%**************************************************************
% Glossario
%**************************************************************
%\renewcommand{\glossaryname}{Glossario}

% CRUD, backend, frontend, REST


\newglossaryentry{umlg}
{
    name=\glslink{uml}{UML},
    text=UML,
    sort=uml,
    description={in ingegneria del software \emph{UML, Unified Modeling Language} (ing. linguaggio di modellazione unificato) è un linguaggio di modellazione e specifica basato sul paradigma object-oriented. L'\emph{UML} svolge un'importantissima funzione di ''lingua franca'' nella comunità della progettazione e programmazione a oggetti. Gran parte della letteratura di settore usa tale linguaggio per descrivere soluzioni analitiche e progettuali in modo sintetico e comprensibile a un vasto pubblico.}
}

\newglossaryentry{restg}
{
	name=\glslink{rest}{REST},
	text=REST,
	sort=rest,
	description={Si tratta di un tipo di architettura software per i sistemi di ipertesto distribuiti come il World Wide Web. Un concetto importante in REST è l’esistenza di risorse (fonti di informazioni), a cui si può accedere tramite un identificatore globale (un URI). Per utilizzare le risorse, le componenti di una rete (componenti client e server) comunicano attraverso una interfaccia standard (ad es. HTTP) e si scambiano rappresentazioni di queste risorse.}
}

\newglossaryentry{phpg}
{
	name=\glslink{php}{PHP},
	text=PHP,
	sort=php,
	description={Linguaggio di scripting interpretato, con licenza open source e libera, originariamente concepito per la realizzazione di pagine web dinamiche.}
}

\newglossaryentry{agileg}
{
	name=\glslink{agileg}{Agile},
	text=Agile,
	sort=Agile,
	description={Insieme di principi per lo sviluppo software sotto i quali requisiti e soluzioni
		evolvono tramite gli sforzi collettivi del gruppo auto-organizzatopolifunzionale. L’obiettivo di questo modello è soddisfare il cliente, rilasciando software in
		maniera continua e in periodi brevi, in modo che possa osservare l’andamento dello
		sviluppo e i risultati ottenuti. Di conseguenza è anche lecito accogliere eventuali
		cambiamenti dei requisiti durante l’intero ciclo di sviluppo.}
}

\newglossaryentry{apig}
{
	name=\glslink{apig}{api},
	text=API,
	sort=Api,
	description={Insieme di servizi (procedure, funzioni, strutture dati) che un sistema espone in qualche modo ai propri utilizzatori.}
}

\newglossaryentry{urlg} {
	name=URL,
	description={
		URL (Uniform Resource Locator) è una sequenza di caratteri che individua univocamente una risorsa web 
		collegata alla rete internet.
	}
}

\newglossaryentry{crudg} {
	name=CRUD,
	description={
		CRUD indica le quattro operazioni fondamentali effettuabili su
		dispositivi di memoriazzazione: creazione (Create), lettura (Read), aggiornamento (Update) e cancellazione (Delete).
	}
}

